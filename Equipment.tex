\section{Detailed study of Equipments}

%\begin{figure}
%  \centering
%  \includegraphics[width=4in]{gecko}
%  \caption[Close up of \species{Hemidactylus}]
%  {Close up of \species{Hemidactylus}, which is part the genus of the gecko family. It is the second most 
%    speciose genus in the family.}
%\end{figure}

%\newtheorem{name}{Printed output}



\subsection{Tank}



large grey plastic vessel

Should we line the inner surface? Stan has been investigating various
non-reflective surfaces

needs investigation into water compatibility and possible emanation





\subsection{Mechanical Gantry System}


As noted before in the introduction, the PTF measurements are made by moving a pair of optical heads to
different positions to illuminate the 20\" PMT.  This motion is accomplished by having the optical heads be on
a pair of moveable gantries.  The gantries provide movement in five different axes: X, Y, Z, rotation and tilt.
The X, Y and Z motion is on three separate linear stages driven by stepper motors,
with the dimensions of the stages being XXXYYY, XXXYYY
and XXXYYY.  The tilt motion is provided by a rotary stage; the tilt motion of the optical head is
provided by a XXXYYY driven by a stepper motor.
This is all shown diagramatically in Figure XXXYYY; these elements are all duplicated for
the two gantries.  The gantry motors are controlled by a Galil 8143 motor controller; the controller is configured
over an ethernet link.  The configuration is handled in a MIDAS program and webpage, described in Section
\ref{Sec:DAQ_Controls}.

Limit switches provide some degree of protection against collisions.  But the two gantries can
be moved into the same space; so additional higher-level collision avoidance has been
programmed into the software control programs.



Various tests have been performed to understand the reproducibility of the gantry motion.  This tests have shown
XXXYYY






\subsection{Optical System/Head}

Shimpei

collimate/polarize the incoming light

watertight

how to manipulate?

new sources?



\subsection{Optical Fibre}
ideal wavelength range is 350--650~nm

consider bending radius

throughput

tested three samples

selected FT200UMT

good transmission across the wavelengths tested~(378--777~nm)

seemed almost unaffected by curvature

fibre must reach several metres from laser through manipulator to the
optical head

how to mount, etc. to prevent tangles, excessive bending, etc.


\subsection{Magnetic Field Compensation}

assess magnetic shielding, magnetic compensation

GIRON magnetic shielding (http://www.lessemf.com/)

GIRON frame

Helmholtz coils, could include the dimensions, wire details,
measured resistance

could describe the testing we have done, eg. setting the voltage and
measuring the field with a handheld gaussmeter

Phidget on each gantry


coil positioning, maybe outside frame supports for coils, precision measurement of coil position

direction of current and maximum external fields

consider when synchrotron is on

control development

might have to do each scan more than once (debugging, etc.)

Magnetic field scans:
\begin{itemize}
\item{\bf Scan 0} no current, no GIRON, calculate the the fields that are needed from the coils
\item{\bf Scan 1} set currents
\item{\bf Scan 2} one layer of GIRON
\item{\bf Scan 3} two layers of GIRON
\end{itemize}


\subsection{Light Tightness/Dark room}

Stan

caulking

air vent light baffles

painting

curtains


\subsection{Water System}
Andy and Peter

Water circulation and filtration system

degassification is a major item


\subsection{DAQ/Controls}
\label{Sec:DAQ_Controls}


collision avoidance is in place and working

collision avoidance in the xz-plane will be necessary once a PMT is in place

further testing and refinement will take about 1 week

how to mount PMTs in the tank




\subsection{Photosensor}

measure different PMTs
\begin{itemize}
\item SK 20 inch
\item IceCube 13 inch, contact Darren Grant
\item hybrid photosensor
\end{itemize}

should contact Japanese, how many PMTs and when?

LBNE has made measurements

angular dependence

magnetic field dependence


measurements before the tank is filled with water

HV and signal must reach from high voltage power supply to the optical
head

